\documentclass[11pt,a4paper]{article}
\usepackage{geometry}
\geometry{a4paper, margin=1in}

\usepackage{titlesec} % For section formatting
\usepackage{lipsum}   % For dummy text
\usepackage{natbib}   % For bibliography
\usepackage{hyperref}

\title{\textbf{Gearbox Failure Prediction in Wind Turbines}}
\author{
    Harshavardhan Reddy Dhoma \\
    \href{mailto:hdhoma@iu.edu}{hdhoma@iu.edu} \\
    \textit{Indiana University} \\
    \and
    Jagadeesh Kovi \\
    \href{mailto:jagakovi@iu.edu}{jagakovi@iu.edu} \\
    \textit{Indiana University}
    \and
    Suraj Vamshi Muthyam \\
    \href{mailto:smuthyam@iu.edu}{smuthyam@iu.edu} \\
    \textit{Indiana University}
}

\date{}


\begin{document}

\maketitle

{\centering{project-jagakovi-hdhoma-smuthyam} \\} % name of your repo goes here
\vspace{1cm} % Add some space before the next content

\section*{Abstract}
This project aims to enhance the reliability and efficiency of wind turbines by minimizing unplanned failures and maximizing power generation. The primary objective is to develop an effective Failure Forecasting model for predicting Gearbox failure in wind turbines. The model will be trained utilizing historical data and will leverage machine learning techniques to consider various influencing factors, including temperature variations, wind direction, and Yaw angle. By doing so, this project seeks to contribute to the sustainability of renewable energy sources, ultimately improving the performance and longevity of wind turbine systems.
% \lipsum[1] % Dummy text for abstract. Replace with your content.

\section*{Keywords}
Wind turbine, Gearbox failure, Failure forecasting model, Renewable energy, Machine learning
% Keyword1, Keyword2, Keyword3, ...

\section{Introduction}
Wind turbines have become essential in our renewable energy landscape, providing clean and sustainable power. However, they often face unplanned failures leading to downtime and reduced energy output. To address this issue, our project focuses on developing a Failure Forecasting model, specifically for predicting Gearbox failures.

Our main goal is to leverage machine learning and historical data to create an accurate model for anticipating Gearbox failures in wind turbines. These failures can significantly disrupt wind energy systems, incurring costly repairs and operational downtime. By predicting these failures, we can implement proactive maintenance strategies, reduce unexpected interruptions, and optimize energy production.

We consider several influencing factors, including temperature variations, wind direction, and Yaw angle, all of which play a vital role in determining the health and performance of wind turbines.

This project not only improves wind turbine efficiency and longevity but also aligns with our commitment to sustainable renewable energy sources. By minimizing unplanned failures and maximizing power generation, we contribute to a cleaner and more eco-friendly energy landscape. This introduction paves the way for a detailed exploration of our project's scope, methodology, and expected outcomes in the subsequent sections.


% \lipsum[2-3] % Dummy text for introduction. Replace with your content.

\subsection*{Previous work} % Dummy text for Previous work. Replace with your content.
The authors at frontier \cite{reference2}, focus on statistical learning-based approaches for fault diagnosis and anomaly detection. The next reference  \cite{reference1} discuss about the effects of vibration on the failure of the gearboxes. The paper \cite{reference3} primarily focuses on experimental methods such as SEM imaging, nanoindentation, and Hertzian stress calculations to analyze the damage and failure modes of the bearings. We refer the mentioned research to develop our ML algorithms.


\section{Methods}
In order to forecast Gearbox failures in wind turbines, a range of machine learning algorithms could be implemented. Naive Bayes is known for its simplicity and efficiency, making it suitable for classification tasks. Support Vector Machines (SVM) are proficient in handling binary classification problems, while k-Nearest Neighbors (KNN) offers a straightforward yet effective approach for limited data scenarios. Decision Trees provide interpretability and feature selection, aiding in understanding predictive factors. Random Forest, with its ensemble learning approach, is adept at capturing complex relationships in data. Logistic Regression, a staple in classification tasks, offers a straightforward and interpretable model development. 

We will select the most appropriate algorithm based on experimentation, guided by the characteristics of the dataset and the specific requirements of predictive maintenance. The optimal approach to forecasting Gearbox failures will be chosen through thorough performance evaluation, ensuring the effectiveness of the selected model.

% \lipsum[5-6] % Dummy text for methods. Replace with your content.

\bibliographystyle{plain}
\bibliography{references} % 'references.bib' should be the name of your BibTeX file

\end{document}
